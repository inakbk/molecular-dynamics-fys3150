\documentclass[11pt,a4wide]{article}
\usepackage{verbatim}
\usepackage{listings}
\usepackage{graphicx}
\usepackage{a4wide}
\usepackage{color}
\usepackage{amsmath}
\usepackage{amssymb}
\usepackage[dvips]{epsfig}
\usepackage[T1]{fontenc}
\usepackage{cite} % [2,3,4] --> [2--4]
\usepackage{shadow}
\usepackage{hyperref}

\setcounter{tocdepth}{2}
\usepackage{url} %url + line break
\newcommand{\dm}[1]{\mathrm{d}#1}

\lstset{language=c++}
\lstset{alsolanguage=[90]Fortran}
\lstset{basicstyle=\small}
\lstset{backgroundcolor=\color{white}}
\lstset{frame=single}
\lstset{stringstyle=\ttfamily}
\lstset{keywordstyle=\color{red}\bfseries}
\lstset{commentstyle=\itshape\color{blue}}
\lstset{showspaces=false}
\lstset{showstringspaces=false}
\lstset{showtabs=false}
\lstset{breaklines}


%lager heftig forside:
\newcommand*{\titleAT}{\begingroup % Create the command for including the title page in the document
\newlength{\drop} % Command for generating a specific amount of whitespace
\drop=0.1\textheight % Define the command as 10% of the total text height

\rule{\textwidth}{1pt}\par % Thick horizontal line
\vspace{2pt}\vspace{-\baselineskip} % Whitespace between lines
\rule{\textwidth}{0.4pt}\par % Thin horizontal line

\vspace{0.5\drop} % Whitespace between the top lines and title
\centering % Center all text
\textcolor{black}{ % Red font color
{\Huge Atomic modeling of argon}\\[0.75\baselineskip] % Title line 1
%{\Large Tema:}\\[0.75\baselineskip] % Title line 2
%{\Huge Lydmåling og hørselstesting} % Title line 3
} 

\vspace{0.25\drop} % Whitespace between the title and short horizontal line
\rule{0.3\textwidth}{0.4pt}\par % Short horizontal line under the title
\vspace{0.25\drop} % Whitespace between the thin horizontal line and the author name

{\Large \textsc{Project 5, FYS-3150\\[0.75\baselineskip] \normalsize{Ina K. B. Kullmann, candidate nr: 20}
}}\par % Author name

%\vfill % Whitespace between the author name and publisher text

\vspace{0.25\drop} % Whitespace between the title and short horizontal line
\rule{0.3\textwidth}{0.4pt}\par % Short horizontal line under the title
\vspace{0.25\drop} % Whitespace between the thin horizontal line and the author name

\begin{abstract}
The aim of this project is to numerically find the critical temperature for the two dimentional Ising model by using the metropolis algorithm. We will first test the implementation of the algorithm carefully, first by comparing with theoretical values calculated for a small system. Then we will see if the algorithm behaves as expected according to our physical intuition for a larger system.

When we have found a estimate for the critical temperature we will compare it to Lars Onsagers analytical result.

\end{abstract}
\vspace*{0.25\drop} % Whitespace under the publisher text

\begin{center}
{ \scriptsize \noindent All source codes can be found at: \url{https://github.com/inakbk/molecular-dynamics-fys3150}. }
\end{center}

\rule{\textwidth}{0.4pt}\par % Thin horizontal line
\vspace{2pt}\vspace{-\baselineskip} % Whitespace between lines
\rule{\textwidth}{1pt}\par % Thick horizontal line

\endgroup}
%kode slutt for heftig forside


\begin{document}
%\maketitle
\titleAT % This command includes the title page


\newpage
\tableofcontents
\newpage

\section{Introduction}

Molecular dynamics (MD) is a computer simulation method used to study atoms and molecule structure and movement. In a MD simulation the atoms or molecules are allowed to interact trough a force given by a potential for a given time. This makes it possible to study the systems development over time. 

MD is a a type of N-body simulation since the simulation often consists of a large number of atoms or molecules. It is therefore possible to use MD to study stastistical properties of a large system consisting of N such atoms or molecules. For systems that obey the ergodic hypothesis the evolution of a single molecular dynamics simulation may be used to determine macroscopic thermodynamic properties of the system. This is because the time averages of an ergodic system correspond to microcanonical ensemble averages\footnote{\url{https://en.wikipedia.org/wiki/Molecular\_dynamics} 3.dec 11:25}.

Often the main motivation to use Molecular dynamics is that it is not possible to determine properties of the system analytically because of the large number of particles. The main limitation for the numerical simulation is the computer recources available, but also cumulative errors in the numerical integration. The first is solved by applying periodic boundary conditions while the latter is solved by proper selection of algorithms and parameters. In this paper we will have a look at two numerical integration methods; the Euler-Cromer method and the Velocity Verlet integrator. 


In this paper we will study the properties of a large system consisting of Argon atoms. And compare with experimental data(?). We will have a constant number of particles, a constant volume and a more or less constant engergy (depending on the integrator). We are more interested in the statistical properties of the system than in the individual motion of each of the particles. We want to sample microstates from the microcanonical ensemble (NVE).
(?se over avsnittet over?)


The applications of MD is many ranging from chemical physics, materials science and the modelling of biomolecules.
What areas of physics can it be used in? Chemistry and biology?(fra oppg)
see 'Areas of application and limitations' at \url{https://en.wikipedia.org/wiki/Molecular\_dynamics} and google




\section{Theory}


\section{Numerical methods}

\subsection{Periodic boundaryconditions (PBCs)}
To avoid problems with boundary effects we will apply periodic boundary conditions. This has a great analogy to 'old' video games such as Snake 2. If the snake head passes through one side of box, it re-appears on the opposite side with the same velocity. In the system of atoms this would mean that if an atom should leave the simulation box at one side it will enter on the oposite side with the same values for the physical parameters as it had before it left. This also implies that an attom at the edge of the box will interact with an atom at the opposite side of the box so that every atom have the same number of 'neighbours'. 

MD simulations that use periodic boundary conditions have a large number of unit cells (define?), a group of atoms, so that a larger system can be created by repeating these cells in space. 

The size of the simulation box must also be large enough to prevent unphysical behaviour. If the box is too small one unit cell (a few atoms) might interact with itself. In the Snake analogy this would mean that the "head" interacts with or bites its own "tail" trough the wall which is allowed in the game, but not very physical. Thus the box size have to be large enough relative to the size of a unit cell, length of the simulation and the desired accuracy\footnote{\url{https://en.wikipedia.org/wiki/Periodic\_boundary\_conditions\#Practical\_implementation:\_continuity\_and\_the\_minimum\_image\_convention}}.

\paragraph{The minimum image convention} (due to calculation of forces)
When an atom leave the simulation box it must re-enter the box on the opposite side. This leads to two possible strategic choices: (A) 'fold back' particles into the simulation box when they leave it, or (B) let them go on into the other side of the box?

....ahh, this is relevant for computing forces/potentials!
An atom which has passed through one face of the simulation box should re-enter through the opposite face-or its image should do it. Evidently, a strategic decision must be made: Do we (A) “fold back” particles into the simulation box when they leave it, or do we (B) let them go on (but compute interactions with the nearest images)? The decision has no effect on the course of the simulation, but if the user in interested in mean displacements, diffusion lenghts, etc., the second option is preferable.

\section{Creating the lattice}




\subsection{The Maxwell-Boltzmann distrubution (giving velocities)}
The atoms are usually given velocities according to the Maxwell-Boltzmann distribution so that
\begin{align}
	\label{eq:maxwell_boltzmann}
	P(v_i)\dm{v_i} = \left(\frac{m}{2\pi k_B T}\right)^{1/2} \exp\left(-\frac{m v_i^2}{2k_B T}\right) \dm{v_i},
\end{align}
where $m$ is the mass of the atom, $k_B$ is Boltzmann's constant and $T$ is the temperature. We recognize this as a normal distribution with zero mean and standard deviation $\sigma = \sqrt{k_B T/m}$. This is a good choice because bla bla bla bla (?) 


But this will result in a nonzero net momentum in the system which we will remove. The momentum in a given direction for one atom is $p_{i} = mv_i$ where the mass is $m$ and $v_i$ is the velocity in direction $i$. The total momentum in one direction, assuming that all particles have the same mass, is given by:
\[
p_{tot} = m(v_{i,1} + v_{i,2} + v_{i,3} + \dots + v_{i,n})
\]
where $n$ is the total number of atoms in the simulation. 

We want to remove this momentum evenly from all the particles. The momentum we want to remove from each particle is then:
\[
p_{rm} = \frac{p_{tot}}{n}
\]
so that $p_i'$ is the momentum for one particle after removing the total momentum:
\[
p_i' = p_i - p_{rm} 
\]
If we write this out
\begin{align*}
mv_i' &= mv_i - \frac{m(v_{i,1} + v_{i,2} + v_{i,3} + \dots + v_{i,n})}{n} \\
\Rightarrow v_i' &= v_i - \frac{v_{i,1} + v_{i,2} + v_{i,3} + \dots + v_{i,n}}{n} \\
&= v_i - \overline{v_i}
\end{align*}
We see that to remove the total momentum we only have to substract the average velocity from each of the particles velocity. 


\subsection{intergration methods}
Euler-Cromer method and the Velocity-Verlet method (discussed in the lecture notes.)
  
  
bla bla bla  
  


\rule{0.3\textwidth}{0.4pt}\par % Short horizontal line under the title  

\end{document}







